\documentclass[10pt]{article}
%\documentclass[10pt,twocolumn]{article}
%\usepackage{fullpage}

% For revision control
%\usepackage{rcs-multi}
%\rcsid{$Id$}
%\rcsid{$Header$}
%\rcskwsave{$Author$}
%\rcskwsave{$Date$} 
%\rcskwsave{$Revision$}
%%\rcsRegisterAuthor{devangel}{Dennis Jos{\'e} Evangelista}
%\rcsRegisterAuthor{devangel}{Dennis J. Evangelista}

% I typically use these
\usepackage{graphicx}
\usepackage{color}
%\usepackage{makeidx}
\usepackage{siunitx}

% PDF metadata
\usepackage{hyperref}
\hypersetup{pdftitle={Aerodynamic stability and control effectiveness of human skydivers during free fall}}
\hypersetup{pdfauthor={G. Cardona, D. Evangelista, Y. Garmashov, N. Ray, K. Tse, D. Wong}}
\hypersetup{pdfsubject={biology}}
\hypersetup{pdfkeywords={biomechanics, evolution, physical model, aerodynamics, gliding, stability, maneuverability}}

% Biology style references
%\usepackage[round]{natbib}
%\setcitestyle{authoryear, round, comma, aysep={;}, yysep={,}, notesep={, }}
%\bibliographystyle{apalike}

% Use Science or PNAS type references.
\bibliographystyle{Science}

% Figures at end for draft
%\usepackage[lists,tablesfirst]{endfloat}

% Title information
\title{Aerodynamic stability and control effectiveness of human skydivers during free fall}
\author{G. Cardona, D. Evangelista*, Y. Garmashov, N. Ray, K. Tse, and D. Wong}
\date{Draft \today}

\begin{document}
\maketitle

\begin{abstract}
We report the aerodynamic stability and control effectiveness of human skydivers in free fall measured using three methods: scaled physical models in a wind tunnel; human skydivers in a vertical wind tunnel; and human skydivers during actual freefall maneuvers.  The effect of posture and movements of the limbs is examined and compared to previously published simulation results and to guidance given during typical skydiving instruction.  Comparison is also made to other animals in freefall and high angle-of-attack aerial maneuvers.  
\end{abstract}

\section{Introduction}
Skydivers as model system for comparative biomechanics.  Maneuvers during flight at high angles of attack are important to understand and potentially relevant to the evolution of aerial behaviors in many taxa.  Many animals in free fall exhibit aerial righting, directed aerial descent, turns, rolls, and controlled safe landings.  Human free fall is an understudied and important point of comparison for several reasons.  Humans use both inertial and aerodynamic mechanisms for maneuvering, depending on the speed.  Skydivers can be asked the rationale behind their techniques and can perform specific test maneuvers.  Humans are also the largest vertebrate known to jump from high altitudes. 

In comparative biomechanics, understanding the role of maneuvers during high angle of attack flight is critical to understanding the evolution of aerial behaviors like flight. We report the aerodynamic stability and control effectiveness of human skydivers in free fall measured using physical models in a wind tunnel. The effect of posture and movements of the limbs is examined and compared to previously published simulation results and to guidance given during typical skydiving instruction, as well as the experience of human skydivers in a vertical wind tunnel and during actual free fall maneuvers. Comparison will also be made to other animals in free fall and in high angle of attack aerial maneuvers.

Animals in free fall must maneuver into preferred stable orientations and to affect safe landings. Human free fall is an under-studied, but important, point of comparison, because humans use both inertial and aerodynamic maneuvering mechanisms (depending on speed); skydivers can be asked the rationale behind techniques and can be asked to perform specific test maneuvers; and because of the important practical applications of skydiving.

Recent studies have highlighted the role of aerial righting and directed aerial descent as part of a continuum of aerial behaviors, for example in gliding ants, gliding stick insects, feathered dinosaurs, geckoes, \emph{Anolis} lizards, \emph{Draco} lizards, gliding snakes, colugos, and flying squirrels (ADD CITATIONS).  These studies illuminate the role of maneuvering, control and stability in gliding at various angles of attack ranging from quite shallow angles (example) to steeper angles more traditionally termed ``parachuting''.  The split between ``parachuting'' and ``gliding'' was arbitrarily fixed to be a steady glide angle of \ang{45} by CITE.  There is no rational basis to do so; an animal in the air typically must accomplish the same tasks, regardless of if it is at a glide angle above or below \ang{45}: reorient using inertial reactions or asymmetric aerodynamic forces, adopt a stable position, and maneuver or glide by producing aerial forces to achieve some desired end position, such as a slow and targeted landing.  By extension, the continuum of aerial behavior would extend to ``powered flight'' as ability to generate forces and control position in the air increases (CITE DUDLEY).  Thus, studies of gliding are likely to be important in understanding how flight evolved, especially studies in comparative phylogenetic contexts, including studies of vastly convergent examples that are phylogenetically unrelated. 

While no flying animal has evolved from humans, data for humans in freefall might be useful as a point of comparison and potential null model.  Furthermore (INSERT practical uses of knowledge about freefall for military or pleasure).  As an experimental system they have the advantage that they can be asked to adopt specific postures, and they can be asked to explain the rationale behind particular maneuvers.  Despite these advantages, published comparative data for parachuting and freefall abilities in humans is strangely lacking. Though a wealth of studies focus on the risk associated with skydiving (CITE THEM) or the public health effects of falls from great heights (CITE THEM), few studies of the actual process of freefall have been published.  (US ARMY STUDY) examined lift and drag coefficients of paratropers, while (JAPANESE GUYS) studied computational models of human skydivers in freefall positions.  

In this study, we address the mechanics of freefall in humans by quantifying static aerodynamic stability and control effectiveness (CITATIONS).  Static aerodynamic stability is a measure of the tendency of a body to experience restoring torques when deflected from an equilibrium position.  Control effectiveness is a measure of the aerodynamic torques applied to the body by movements of control surfaces such as limbs.  Two approaches are adopted here from typical biomechanical studies of other organisms:  physical models in a wind tunnel (CIATIONS) and full-scale measurements on live organisms using kinematics (CITATIONS) or accelerometery (CITATIONS).  

\section{Scaled physical models of skydivers}

Maybe do this second? 

Numerous studies have quantified aerodynamic stability in other taxa (CITE OURSELVES HERE).  Here we follow similar methods (see Section~\ref{sec:Methods} for details).  x-scale physical models were constructed and mounted on a six-axis force and torque sensor in a wind tunnel.  The model was positioned into skydiving postures (CITE SKYDIVER BOOK) to examine stability and control effectiveness of arm and body movements.  

\begin{figure}
\caption{Physical model and sign conventions for aerodynamic data.} 
\end{figure}

Due to wind tunnel limitations we were not able to achieve Reynolds similarity with actual human skydivers.  Over this range, changes in aerodynamic coefficients with Reynolds number are expected to be small (CITATIONS).  Furthermore, we performed full-scale tests of human skydivers, both in a vertical wind tunnel and during actual skydives.  Could show plot too?


\subsection{Stability}
Postures, pitch roll and yaw.  How compare with the Japanese theoretical model?

\begin{figure}
\caption{Stability data.  Maybe do table too. }
\end{figure}

\subsection{Control effectiveness} 
Different movements, pitch roll and yaw.  How compare with the Japanese theoeretical models? 

\begin{figure}
\caption{Control effectiveness data.  Maybe do table too. }
\end{figure}

\section{Full-scale measurements on skydivers in freefall}
Or is it better if we discuss this first? 

\subsection{Vertical wind tunnel measurements}
I think here we want to show a series of rocking tests, yaw and roll maneuvers and use them to estimate stability and control effectiveness. 

\subsection{Actual skydives}
Here we want to test those maneuvers that we can't do in the vertical wind tunnel - delta maneuvers, movement ahead, backwards, and side, as well as yaw and roll.  Maybe not so much rocking tests.  Dieudonne spiral? 

\section{Results}
Preliminary results from model tests agree with non-quantitative skydiver self-assessments of stability and maneuverability. Different skydiving postures exhibit clear differences in stability and in stable orientation relative to flow (Figures 2 and 3).

These results suggest that skydivers maneuver principally using aerodynamic forces vice inertial reactions. At low speeds, zero angular momentum turns affected by inertia of limb movement, changes in body inertia and body position changes drive changes in body orientation (e.g., gymnastics \cite{Plater:1994}, cliff diving, or other acrobatic maneuvers performed at low speed close to the ground). In contrast, at skydiving speeds (\SI{54}{\meter\per\second}, \SI{120}{mph}), maneuvers are dominated by aerodynamic torques, which scale as $\frac{1}{2}\rho u^2\lambda S$. For example, in track/delta posture, elevation of the arms has sufficient control to create a one-for-one angular pitch change of the body (Figure 3).

As a further test of the relative roles of inertial reactions versus aerodynamic forces, we are currently comparing predicted turn and roll dynamics to those observed in the absence of flow (while statically hanging from a line on the ground), and to maneuvers in a full-scale vertical wind tunnel (iFly; Union City, CA) and during actual skydives (Bay Area Skydiving; Byron, CA). Skydivers have recently adopted miniature GPS loggers originally developed for do-it-yourself unmanned aerial vehicles, and regularly use such tracks to examine their own glide performance and compete with others. Using off-the-shelf components (SparkFun; Boulder, CO), we have ground-tested loggers that record accelerations, angular velocities, and magnetometer readings at 50 Hz and GPS positions at 4 Hz (example data, Figure 4). A Kalman filter is then applied to estimate full-scale aerodynamic forces and moments during typical maneuvers, such as the turns and rolls required for entry-level skydiving licensing (US Parachuting Association A- level) [1].

\begin{figure}
\begin{center}
\includegraphics{../figures/asb-poster-figure2/Cmvsaoa-annotated.pdf}
\end{center}
\caption{Nondimensional pitch coefficient ($C_M$) \cite{McCay:2001, McCormick:1976} as a function of angle of attack for typical skydiving body postures shows clear differences in static stability ($\partial C_m/\partial\alpha$); spider posture is unstable and track posture is stable at a lower angle of attack.}
\label{fig:2}
\end{figure}

\begin{figure}
\begin{center}
\includegraphics{../figures/asb-poster-figure3/pitch-control-authority-annotated.pdf}
\end{center}
\caption{$C_M$ as a function of angle of attack for track posture as arm angle is varied \ang{20} up and down, illustrating control effectiveness ($\partial C_m/\partial\delta$) of symmetric arm movements in pitch.}
\label{fig:3}
\end{figure}



\section{Discussion}
Here we should address how humans compare to other animals and to aircraft and any pithy neat things we can figure out. 





\section{Methods and materials}
\label{sec:Methods}
For the high impact journals, method details are buried at the end. 

\subsection{Tests of a six-inch model in a wind tunnel}
Physical models of human skydivers (Figure 1) were constructed using six-inch artists� anatomical manikins (Dick Blick; Galesburg, IL) placed in typical human skydiving postures cite1. Aerodynamic forces (lift, drag, and side force) and moments (pitch, roll, and yaw) acting on models in a wind tunnel were measured using a six degree-of- freedom force/torque sensor (ATI Industrial Automation; Apex, NC). Forces and moments were normalized to the planform area of a ``flat'' resting human. To quantify static aerodynamic stability, models were placed at varying pitch, roll, and yaw angles and the restoring torques acting about the center of gravity were measured and used to obtain static stability coefficients (e.g., $\partial C_m/\partial\alpha$) \cite{McCay:2001, McCormick:1976}. Similarly, control effectiveness was measured by placing limbs in turn or roll positions (see Figure 1) and measuring the resulting yawing or rolling moments to obtain control effectiveness coefficients (e.g., $\partial C_m/\partial\delta$) \cite{McCay:2001, McCormick:1976}.

We compared stability and control effectiveness observed in model tests to statements in skydiving training literature cite1, photo and video of stable positions openly reported on the Internet, and descriptions from interviews with professional skydiving instructors.

\begin{figure}
\begin{center}
\includegraphics{../figures/asb-poster-figure1/skydiver-model.pdf}
\end{center}
\caption{Yaw model of skydiver, constructed using artists' manikin placed in typical human skydiving posture \cite{Poynter:2007}. Aerodynamic forces (lift, drag, and side force) and moments (pitch, roll, and yaw) \cite{McCay:2001, McCormick:1976} were measured using a six degree of freedom force/torque sensor (ATI; Apex, NC). Asymmetric body twist creates a left yaw moment. Similar models created for pitch and roll. }
\label{fig:models}
\end{figure}

\subsection{Full scale tests in an indoor skydiving vertical wind tunnel}
Indoor skydiving tests validate six-inch model results for full scale Re.
Tests also performed on stationary turntable; inertial yawing movements effective at \SI{0}{\meter\per\second} cause severe rolls at \SI{54}{\meter\per\second}.
Stability shapes behavior. 
Subtle movements effective for changing yaw (marginally stable); larger movements needed to change pitch or roll.

\begin{figure}
\begin{center}
\includegraphics{../figures/asb-poster-figure4/figure4.pdf}
\end{center}
\caption{A-C. Volunteer test subject instrumented with wearable accelerometer (SparkFun; Boulder, CO; mass \SI{30}{\gram}) performing yaw maneuver using slight body twist and hand movements.  D. Roll, pitch, and yaw data during the maneuver from Kalman-filtered \SI{50}{\hertz} accelerometer data.  Subject performs \ang{360} left yaw turn, \ang{360} barrel roll, and \ang{360} right yaw turn in within \SI{8}{\second}.}
\label{fig:4}
\end{figure}

\subsection{Data collection during actual skydives from 13,000 feet}
Data collection during \textbf{actual skydives} is underway.
Replicate maneuvers from vertical wind tunnel; also test track posture.
Sensor suite includes \SI{50}{\hertz} accelerometer, \SI{10}{\hertz} magnetometer, \SI{4}{\hertz} GPS position, and 60 fps HD video, tested during tandem skydives.

\begin{figure}
\begin{center}
\includegraphics{../figures/asb-poster-figure5/figure5taller.pdf}
\end{center}
\caption{A.\ Volunteer test subjects prior to jump, inset shows GPS/IMU device (DIYDrones; San Diego, CA; mass \SI{228}{\gram}).  B.\ During tandem skydive from 13,000 feet (\SI{3962}{\meter}), opening at 4,500 feet (\SI{1372}{\meter}).  C.\ GPS track during tandem skydive, view looking NE towards San Francisco Bay.}
\label{fig:5}
\end{figure}




\section{Acknowledgements}
We thank E.\ Buchanan and T.\ Costanza of iFly SF, M.\ Cooper and G.\ Peek for technical advice, and the Berkeley Biomechanics group; R.\ Dudley, E.\ Chang-Siu and Y.\ Munk. We thank the Berkeley Center for Integrative Biomechanics Education and Research for a force sensor.  We dedicate our work to Alex Lowenstein, who inspired this research and whose loss helped us decide that today is a good day to skydive.


\bibliography{../common/parachute}
\end{document}
